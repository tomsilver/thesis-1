\begin{savequote}[75mm]
We can only see a short distance ahead, but we can see plenty there that needs to be done.
\qauthor{Alan Turing}
\end{savequote}

\chapter{Conclusion}

In this thesis, LRS has been defined, characterized, stress tested, contextualized, and implemented. I have argued that the system offers a practical, accessible, generalizable, continuous, and robust benchmark for AI. I have shown how the Luna Game fits into the broader context of existing AI research and proposed the Luna Rating Prediction problem for future study. I have demonstrated with a web-based proof-of-concept that LRS can recruit and sustain large numbers of human players who ask questions of every variety and answer with depth. But the most important question pertaining to LRS remains: will it be used? The ultimate value of LRS is not something that can be theoretically proved or simulated. As a proposed practical test for AI, its merit can only be determined in practice. The true test of the Luna Rating System is yet to come. 