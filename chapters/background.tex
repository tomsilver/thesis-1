% Background section
\chapter{Background}
\label{Background}

Much of the recent history of genome editing consists of building the site-specificity of a nuclease through engineered protein-DNA interactions. Though this technique has proven effective and useful, as seen in the examples of Zinc-Finger Nucleases (ZFNs) and Transcription Activator-Like Effector Nucleases (TALENs), these platforms are difficult to use because of the complexity of programming protein-DNA specificity. In recent years, the CRISPR-Cas bacterial immune system has been extensively used as a genome engineering alternative that relies on RNA-DNA complementarity as the homing interaction that guides the specificity of the Cas9 nuclease. The ease of using such a platform to cleave specific sites in the genome simply by choosing a guide RNA sequence complementary to the target sequence has lead to an explosion in its use, with enormous amounts of interest and capital being contributed to the investigation of further development and application of this technology (Doudna and Charpentier \textit{et al.} 2014).\\

The molecular basis of Cas9 specificity comes from two independent RNA sequences: the CRISPR RNA (crRNA) and the transactivating CRISPR RNA (tracrRNA). In the wild, the tracrRNA and crRNA complex at a region of complementarity, are processed by endogenous nonspecific exonucleases, and then associate with Cas9 for a final crRNA-tracrRNA-Cas9 complex. When using the system in the lab, the crRNA and the tracrRNA are expressed as a single guide RNA (gRNA) strand to simply the formation of the final RNA-Protein complex. Cas9 then scans genomic DNA for a short Protospacer-Adjacent Motif (PAM) sequence, which is hard-coded into the protein. Once a match is found, Cas9 then alters its conformation to check for complementarity between the 20 nucleotide gRNA spacer sequence and the adjacent genomic DNA. If a match is found, Cas9 then makes a double stranded break at that locus (Jinek \textit{et al.} 2012, Anders \textit{et al.} 2014).

Once a double stranded break occurs in a eukaryotic cell, cellular machinery quickly repairs the break through either non-homologous end joining (NHEJ) or homology directed repair (HDR). In the case of NHEJ, random blunt end resections or small nucleotide insertions can occur, causing a deletion or insertion of one or more nucleotides in the final repaired site. This is phenomenon is often used to induce frame-shift mutations into a gene to  knock it out. In the case of HDR, a donor strand of DNA is used to template a precise insertion at the site of repair (Doudna and Charpentier 2014).

Perhaps one of the largest questions in the CRISPR-Cas9 field is that of targetability. Since there is a PAM sequence requirement for all known Cas9 proteins, each can only target a limited subset of all possible genomic targets. Interestingly, since the CRISPR-Cas system evolved as a bacterial immune system against viral DNA, and since different bacterial strains are attacked by vastly different phages, the Cas9 proteins from different strains seem to have differing PAM specificities. Thus, the targeting range of the CRISPR-Cas platform can be expanded by adding more Cas9s to choose from (Kleinstiver \textit{et al.} 2015).

Of similar import is specificity. Because RNA and DNA can form complementary complexes even in the presence of a small number of mismatches between the two molecules' sequences, CRISPR-Cas9 often cleaves at off-target locations in a generally unpredictable manner. This has huge implications in applications of CRISPR-Cas9, as off-target activity can possibly have catastrophic effects on a cell. We have recently described an assay for detecting off-target cleavage events by Cas9 in an unbiased manner, which can be used to better understand the specificities of different Cas9 proteins (Tsai \textit{et al.} 2015).


\newpage

\section{Selected Cas9 Proteins}

\begin{table}[h!]
\centering
\label{my-label}
\begin{tabular}{|l|l|l|}
\hline
{\bf Strain}                                        & {\bf CRISPR System Type} & {\bf Putative PAM} \\ \hline
{\it Lactobacillus buchneri}                   & Type II-A                & 5'-AAAA-3'         \\ \hline
{\it Neisseria meningitidis}                & Type II-A                & 5'-NNNNGATT-3'     \\ \hline
{\it Francisella novicida}                     & Type II-B                & 5'-NG'3'           \\ \hline
{\it Campylobacter jejuni}                          & Type II-C                & 5'-NNNNACA-3'      \\ \hline
{\it Pasteurella multocida} & Type II-C                & 5'-GNNNCNNA-3'     \\ \hline
\end{tabular}
\caption{Selected Cas9 proteins and their strains of origins and putative (predicted) PAM sequences.}
\end{table}

The five orthogonal Cas9 proteins selected for characterization are from a variety of families and CRISPR system subtypes, and have very disparate putative PAM sequences. All orthologs were chosen from the phylogenetic tree of Cas9 orthologues presented in the supplemental documents in  Fonfara \textit{et al} (2013). Note the inclusion of \textit{N. meningitidis} Cas9, which has been profiled previously in the literature, and will be used in part as a positive control. It should also be noted that all selected Cas9s have been shown to have \textit{in vitro} activity with a dual-RNA (non-chimeric) guide in the supplemental information of Fonfara \textit{et al} (2013).


\section{Experimental Approach}

\subsection{Prediction of RNA Guides}
The first step in using any Cas9 is to determine its crRNA and tracrRNA to construct a chimeric gRNA to guide the Cas9 activity. In general (for type II CRISPR systems), the Cas9 open reading frame, CRISPR repeats, and tracrRNA are all very near one another, generally within a single $\approx 5kb$ stretch. The location of these loci can easily be found through a simple BLAST search of SpCas9 against the bacterial genome to find the orthogonal Cas9 gene. Then, a repeat masking algorithm is applied to find fixed-length repeats in the region $\pm 2kb$ from the Cas9 protein. These repeats are the CRISPR repeats, which are spaced by fixed-length spacer sequences. Finally, the crRNA repeats are aligned against the same $\pm 2kb$ region to find the tracrRNA, which starts at the region of complementarity and ends at a poly-T transcriptional termination signal. Once these two sequences are found, they are fused together via a GAAA linker loop to form a single gRNA. This method was used computationally implemented to predict (with success) the gRNA for SaCas9 in Kleinstiver \textit{et al.} 2015. The same pipeline will be used to predict the crRNA, tracrRNA, and gRNA of the orthogonal Cas9 proteins in this study.



\subsection{Positive Selection}

To test if the predicted gRNA can indeed transactivate and guide the orthogonal Cas9 protein, a positive selection assay will be used. In this assay, a toxic gene is placed in front of a glucose-inducible promoter on a plasmid containing a set Cas9 target site containing the predicted PAM sequence. The plasmid is co-transformed with a plasmid containing both the orthogonal Cas9 gene and the gRNA gene and the bacteria are grown in glucose-containing media. If there is bacterial growth, then the gRNA activated and guided Cas9 to cleave the target site on the toxic plasmid-containing gene, keeping the bacterium alive. Thus, this positive selection experiment can be used as an early screen for gRNA efficiency.

\subsection{Negative PAM Depletion Assay}

Once the gRNA is confirmed to work through the positive selection assay, a negative selection assay will be used to fully characterize the PAM specificity of the protein. In this assay, an antibiotic resistance gene is placed on the same plasmid as a fixed target site followed by a randomized 8bp PAM region (i.e. NNNNNNNN). This plasmid is co-transformed with a plasmid containing the orthogonal Cas9 and gRNA, and the bacteria are grown in antibiotic-containing media. If the randomized PAM matches the Cas9 PAM, then the resistance plasmid is cleaved, and the bacterium dies. At the end of the assay, the region containing the spacer and PAM is sequenced using high-throughput sequencing. The relative abundance of each PAM sequence is analyzed computationally. PAM sequences that are the most depleted are the most active PAM sequences for the Cas9 protein. An example of the results from this assay, as found in Kleinstiver \textit{et al.} 2015 can be found in Figure 1. The code used to determine relative PAM frequencies can be found in the supplemental information of that same text.

\begin{figure}
\begin{center}
\begin{tabular}{ | l | l | l | }
\hline
	\textbf{PAM} & \textbf{Count} & \textbf{Frequency} \\ \hline
	GTT & 27360 & 3.00391958805898E-2 \\ \hline
	TGT & 26445 & 2.9034595579758699E-2 \\ \hline
	TTT & 24614 & 2.70242970542704E-2 \\ \hline
	GGT & 23149 & 2.5415838649114501E-2 \\ \hline
	TAT & 22106 & 2.42707040985496E-2 \\ \hline
	TGC & 21063 & 2.3125569547984798E-2 \\ \hline
	AGT & 20840 & 2.2880732534776699E-2 \\ \hline
	GCT & 20368 & 2.2362512488883501E-2 \\ \hline
	\vdots & \vdots & \vdots \\ \hline
	TAG & 6433 & 7.0629439729471598E-3 \\ \hline
	CAG & 4531 & 4.9746928558096603E-3 \\ \hline
	AAG & 866 & 9.5080203335492597E-4 \\ \hline
	GAG & 457 & 5.0175118850254196E-4 \\ \hline
	GGG & 386 & 4.2379859685335002E-4 \\ \hline
	TGG & 322 & 3.5353147198647402E-4 \\ \hline
	AGG & 309 & 3.39258462247889E-4 \\ \hline
	CGG & 278 & 3.0522282364049599E-4 \\ \hline
\end{tabular}
\caption{Table Excerpt of PAM Depletion Assay analysis for EGFP site 1 for wild-type \textit{S. pyogenes} Cas9. Cells are in descending frequency order. Note that the most depleted PAM sequences are those in the format NGG, and are depleted by two orders of magnitude. The second most depleted set are those in the format NAG, which are less depleted but still somewhat depleted compared to the rest of the PAM sequences.}
\end{center}
\end{figure}

\newpage
\subsection{GUIDE-Seq Specificity Profiling}
GUIDE-Seq is a method for unbiased detection of off-target nuclease activity that we recently described in Tsai \textit{et al.} 2014. In this assay, a short, end-stabilized DNA strand is captured into double stranded breaks through NHEJ capture, wherein the blunt ends of the oligo is captured into the blunt double stranded break by endogenous DNA repair machinery. These regions are then linearly amplified off of the known oligo sequence and sequenced with high-throughput sequencing. By mapping these sequences back to the genome, the sites of off-target cleavage events are elucidated.

In this study, GUIDE-seq will be performed for each orthogonal Cas9 for a number of targetsites to determine the number of off-target cleavage events that each protein is generally responsible for. This will provide a set of data that will be valuable in determining which Cas9 is best used in a given application (e.g. a Cas9 with very low off-target effects may be better for therapeutic applications).

\subsection{Expected Results and Potential Roadblocks}

The methods outlined above are, in a way, a ``Hail Mary" approach to test the activity of the selected Cas9s. For the positive selection to work, three things must simultaneously be successful. First, both the chimeric guide and the Cas9 must both be successfully expressed in high numbers, and be non-toxic to the cell. Second, the predicted crRNA and tracrRNAs must be correct, and the chimeric guide constructed from them must activate and guide the Cas9. Third, the Cas9 must induce a double stranded break in the targeted DNA sequence. Only then will the plasmid containing the toxic gene be degraded, and the cell will live. To ensure that the system should work in its totality, the \textit{N. meningitidis} Cas9 (NmCas9) will be used as a positive control, as it has been profiled in a number of published studies, and so its activity in the positive selection can be expected. In the case of further complications, individual steps of the expression and cleavage process will be evaluated. For example, the expression of the protein and the gRNA could be evaluated via RT-qPCR. One other possible concern is whether or not the tracrRNA secondary structure in the gRNA will accurately reflect the endogenously processed tracrRNA that would ordinarily transactivate the Cas9 protein. For this reason, multiple different truncations of the tracrRNA will be predicted and tested. Though we expect at least some nonempty subset of the 5 selected Cas9s to work in this assay, it is possible that only the NmCas9 will work successfully, and that the other four Cas9s simply do not work \textit{in vivo} outside of their host organisms.

The negative PAM depletion assay also carries a few assumptions that must be taken into consideration. In particular, the random 8bp PAM is assumed to be fully random, with every sequence having an equal probability of appearing. However, due to inherent biases in the oligonucleotide synthesis process, this is not always indeed the case. To account for this, the PAM depletion assay will be carried out with the same randomized PAM library with \textit{S. pyogenes} Cas9, which has the most well-characterized PAM sequence. In doing so, we can make sure that there are not significant biases in the PAM library that may skew our PAM characterization results.

If we show that one or more of these Cas9s have activity \textit{in vivo}, then the overall sequence specificity of the Cas9 can be profiled in an unbiased fashion through GUIDE-Seq. Though we have shown previously that off-target profiling cannot be theoretically or computationally predicted (Tsai \textit{et al.} 2015), we hypothesize that Cas9s with shorter putative PAM sequences (e.g. \textit{F. novicida}, which has a putative PAM of 5'-NG-3') will have less stringent sequence specificity.