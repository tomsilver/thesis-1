
%!TEX TS-program = xelatex
%!TEX encoding = UTF-8 Unicode

\documentclass{Dissertate}

\begin{document}

% the front matter
% Some details about the dissertation.
\title{Luna: A Game-Based Rating System for Artificial Intelligence}
\author{Tom Silver}
\advisor{Stuart M. Shieber}

% ... about the degree.
\degree{Artium Baccalaureus (A.B.)}
\field{Computer Science and Mathematics}
\degreeyear{2016}
\degreemonth{May}
\department{Computer Science and Mathematics}

% ... about the candidate's previous degrees.
%\pdOneName{B.S.}
%\pdOneSchool{Boston University}
%\pdOneYear{2018}
%
%\pdTwoName{M.A.}
%\pdTwoSchool{Monster's Univeristy}
%\pdTwoYear{2021}

\maketitle
%\copyrightpage

% Force all citations to show up in references
\nocite{*}

\setstretch{1.2}
\abstractpage
\tableofcontents
%\listoffigures
\dedicationpage
\acknowledgments

\doublespacing

% include each chapter...
\setcounter{chapter}{-1}  % start chapter numbering at 0

\chapter{Introduction}
\label{introduction}

Much of the recent history of genome editing consists of building the site-specificity of a nuclease through engineered protein-DNA interactions. Though this technique has proven effective and useful, as seen in the examples of Zinc-Finger Nucleases (ZFNs) and Transcription Activator-Like Effector Nucleases (TALENs), these platforms are difficult to use because of the complexity of programming protein-DNA specificity.
In recent years, the CRISPR-Cas bacterial immune system has been extensively used as a genome engineering alternative that relies on RNA-DNA complementarity as the homing interaction that guides the specificity of the Cas9 nuclease. The ease of using such a platform to cleave specific sites in the genome simply by choosing a guide RNA sequence complementary to the target sequence has lead to an explosion in its use, with enormous amounts of interest and capital being contributed to the investigation of further development and application of this technology (Doudna and Charpentier \textit{et al.} 2014).

\chapter{Background}
\label{Background}

\section{Test Section}
\subsection{Test Subsection}
Much of the recent history of genome editing consists of building the site-specificity of a nuclease through engineered protein-DNA interactions. Though this technique has proven effective and useful, as seen in the examples of Zinc-Finger Nucleases (ZFNs) and Transcription Activator-Like Effector Nucleases (TALENs), these platforms are difficult to use because of the complexity of programming protein-DNA specificity.
In recent years, the CRISPR-Cas bacterial immune system has been extensively used as a genome engineering alternative that relies on RNA-DNA complementarity as the homing interaction that guides the specificity of the Cas9 nuclease. The ease of using such a platform to cleave specific sites in the genome simply by choosing a guide RNA sequence complementary to the target sequence has lead to an explosion in its use, with enormous amounts of interest and capital being contributed to the investigation of further development and application of this technology (Doudna and Charpentier \textit{et al.} 2014).

%\chapter{Introduction}
\label{introduction}

Much of the recent history of genome editing consists of building the site-specificity of a nuclease through engineered protein-DNA interactions. Though this technique has proven effective and useful, as seen in the examples of Zinc-Finger Nucleases (ZFNs) and Transcription Activator-Like Effector Nucleases (TALENs), these platforms are difficult to use because of the complexity of programming protein-DNA specificity.
In recent years, the CRISPR-Cas bacterial immune system has been extensively used as a genome engineering alternative that relies on RNA-DNA complementarity as the homing interaction that guides the specificity of the Cas9 nuclease. The ease of using such a platform to cleave specific sites in the genome simply by choosing a guide RNA sequence complementary to the target sequence has lead to an explosion in its use, with enormous amounts of interest and capital being contributed to the investigation of further development and application of this technology (Doudna and Charpentier \textit{et al.} 2014).

In what is perhaps the most important paper in the field, the Doudna (UC Berkeley) and Charpentier (Umea University, Sweden) labs teamed up in 2012 to fully characterize the homing mechanism of the Cas9 nuclease. Through exclusively \textit{in vitro} assays, they found that Cas9, the tracrRNA, the crRNA with a spacer specifying the correct target, and magnesium ions, are all that is needed for cleavage of target DNA to occur. They further validated that complementarity between the spacer and the target is necessary for cleavage, and that one or two mismatches in this spacer::target complementarity is tolerated. Additionally, they showed that cleavage of a target depends on a ``protospacer adjacent motif" (PAM) sequence, which is specified by the Cas9 protein, not the spacer (Jinek and Chylinski \textit{et al.} 2012).

\begin{figure}[!h]
	\begin{center}
	\centerline{
	\includegraphics[width=1.15\textwidth]{figures/grna.png}
	}
	\caption{An RNA complex guides Cas9 to its target. \textbf{(a)} The matured crRNA and tracrRNA complex to guide Cas9. \textbf{(b)} A single gRNA comprising of the crRNA, a linker loop, and a crRNA guides Cas9 with the same efficiency as the dual-guide system. (Adapted from Jinek and Chylinski \textit{et al.} 2012)}
	\label{fig:grna}
	\end{center}
\end{figure}

Finally, realizing that the CRISPR-Cas system could be an easily programmable nuclease, the authors constructed a synthetic platform that marked the first time that this system was engineered for easier biotechnological use. They fused the tracrRNA and crRNA sequences (as they would be after maturation by RNAseIII) together with a short linker loop and expressed the entire construct as a single sequence. This led to very strong cleavage, indicating that bypassing the maturation step increases the activity of CRISPR-Cas restriction. This fused RNA molecule concept, dubbed the guide RNA (gRNA) is now nearly universally used in all applications of the CRISPR-Cas platform.

The findings of Jinek and Chylinski \textit{et al.} (2012) kicked off a frenzy of development for the CRISPR-Cas9 platform, but also made researchers more curious about the nature of this system and how it is used by bacteria in the wild. Now that the mechanisms of specificity and restriction had been elucidated, the last question was how the system adapted to pathogens through the integration of new spacer sequences in the CRISPR array.

%
%The molecular basis of Cas9 specificity comes from two independent RNA sequences: the CRISPR RNA (crRNA) and the transactivating CRISPR RNA (tracrRNA). In the wild, the tracrRNA and crRNA complex at a region of complementarity, are processed by endogenous nonspecific exonucleases, and then associate with Cas9 for a final crRNA-tracrRNA-Cas9 complex. When using the system in the lab, the crRNA and the tracrRNA are expressed as a single guide RNA (gRNA) strand to simply the formation of the final RNA-Protein complex. Cas9 then scans genomic DNA for a short Protospacer-Adjacent Motif (PAM) sequence, which is hard-coded into the protein. Once a match is found, Cas9 then alters its conformation to check for complementarity between the 20 nucleotide gRNA spacer sequence and the adjacent genomic DNA. If a match is found, Cas9 then makes a double stranded break at that locus (Jinek \textit{et al.} 2012, Anders \textit{et al.} 2014).
%
%Once a double stranded break occurs in a eukaryotic cell, cellular machinery quickly repairs the break through either non-homologous end joining (NHEJ) or homology directed repair (HDR). In the case of NHEJ, random blunt end resections or small nucleotide insertions can occur, causing a deletion or insertion of one or more nucleotides in the final repaired site. This is phenomenon is often used to induce frame-shift mutations into a gene to  knock it out. In the case of HDR, a donor strand of DNA is used to template a precise insertion at the site of repair (Doudna and Charpentier 2014).
%
%Perhaps one of the largest questions in the CRISPR-Cas9 field is that of targetability. Since there is a PAM sequence requirement for all known Cas9 proteins, each can only target a limited subset of all possible genomic targets. Interestingly, since the CRISPR-Cas system evolved as a bacterial immune system against viral DNA, and since different bacterial strains are attacked by vastly different phages, the Cas9 proteins from different strains seem to have differing PAM specificities. Thus, the targeting range of the CRISPR-Cas platform can be expanded by adding more Cas9s to choose from (Kleinstiver \textit{et al.} 2015).
%
%Of similar import is specificity. Because RNA and DNA can form complementary complexes even in the presence of a small number of mismatches between the two molecules' sequences, CRISPR-Cas9 often cleaves at off-target locations in a generally unpredictable manner. This has huge implications in applications of CRISPR-Cas9, as off-target activity can possibly have catastrophic effects on a cell. We have recently described an assay for detecting off-target cleavage events by Cas9 in an unbiased manner, which can be used to better understand the specificities of different Cas9 proteins (Tsai \textit{et al.} 2015).
%\begin{savequote}[75mm]
Don't get fooled by people who claim to have a solution to Artificial General Intelligence... Ask them what error rate they get on MNIST or ImageNet.
\qauthor{Yann LeCun}
\end{savequote}


\chapter{The Luna Rating System}

\section{Evaluating the Intelligence of Machines}

\subsection{Introduction}

Research in artificial intelligence is driven by standardized tests and competitions. The level of success of an algorithm on a widely used test can determine the amount of funding and attention from academia that it receives. With such a burden placed on these tests, one would hope that they provide an accurate reflection of the extent to which an algorithm has achieved artificial intelligence. Unfortunately, existing tests fall far short of this mark \cite{shieber15}. At best, current contests evaluate the performance of specialized algorithms on restricted problem domains that are not guaranteed to generalize, such as object recognition \cite{lecun98, russakovsky15} or robot soccer \cite{anderson11, kitano97}. At worst, tests prematurely claim to be the ultimate arbiter of general intelligence \cite{loebner03}, inspiring a distracting and unwarranted media frenzy when they are passed \cite{anonymous14, shieber14}. In all cases, the current benchmarks for evaluating AI machines are leading the research community away from the creation of truly general intelligence.

In this thesis, I put forth a new system for evaluating general AI. The system invites humans and machines to participate in two-player games in which each player assesses the intelligence of her opponent. The aggregation of these assessments is used to assign a rating to the player. A player's rating is a reflection of the player's demonstrated intelligence. In this sense, the ratings that emerge from the system are the results of a never-ending test of each of the player's general AI. The definition of intelligence is not presupposed in this system; instead, it emerges from the collective judgement of all players. Thus the system is a microcosm of the natural process that humans use to evaluate each other's intelligences. I will argue that the system offers a bright light on the dim path towards machine intelligence.

\subsection{Principles of a Practical Test for Intelligence}

The system proposed in this thesis is meant to steer AI research towards the creation of general intelligence. Therefore, the system must actually be used; a thought experiment will not suffice. Here I present five principles that must be obliged if the test for intelligence is to fulfill its stated purpose.

\subsubsection{Accessibility}
To serve as a useful guide, the test for AI must be constantly accessible to researchers. Ideally the test should be efficient enough that it may be used several times throughout the course of an AI developer's day. This principle discourages a centralized competition that is only held at regular intervals, and instead favors a rating system that can be accessed through the Internet and then carried out using the resources of an average computer.

\subsubsection{Generalizability}
A test need not span all possible areas of intelligence, but the results should reflect the subject's ability to perform in all areas. In this sense, the test should be \textit{AI-complete} --- a machine that does well on this test should do similarly well on any other reasonable test of intelligence. The proposition of the Turing Test, which continues to be held by many researchers, is that the problem domain of natural language is AI-complete. The system proposed here shares this premise.

\subsubsection{Continuity}
Every candidate for AI should be able to observe changes in performance over the course of development. A test that only reports a binary outcome --- pass or fail --- will not be useful for researchers who are not yet close to passing. The test's outcome should instead be continuous, indicating clearly when progress is being made.

\subsubsection{Dependent on People, Independent of Persons}
A fair test should not rely on any one person's interpretation of intelligence. At the same time, intelligence is a social construct that cannot be reasonably appraised without input from humans. Therefore a test for intelligence must take into account the popular conception of intelligence, but it cannot rely on any single human judge to determine its outcome.

\subsubsection{Immunity to Gaming}
It should not be possible for a researcher to ``game the system'' and achieve outsized results by exploiting structure in the test. This principle precludes any sort of hard-coding of knowledge that is \textit{a priori} known to be important for the test. For example, a standardized test fails the Immunity to Gaming principle, since any researcher who observes the results of the test once would be able to submit a machine with the memorized knowledge necessary to pass the test. 

\section{Existing Contests for Artificial Intelligence}

The history of testing AI is long but sparse. It begins in 1950 with the introduction of the Turing Test and meanders into the present day with specialized tests like ImageNet and Robocup. Only recently has there been interest in designing new tests that are more appropriate for practically evaluating general AI \cite{you15}. This interest often manifests as an appeal to move ``beyond the Turing Test'' \cite{marcus14}. However, as I argue below, the Turing Test has never been a practical test for general AI, nor was it ever meant to be. Thus the recent wave of interest is really a call for the first ever practical test for general AI.

\subsection{The Turing Test}

The first formal test for machine intelligence was articulated by Alan Turing, the father of computer science and one of the progenitors of AI \cite{turing50}. Six decades later, the eponymous Turing Test remains at the center of the dialogue surrounding machine intelligence. The Test requires three rooms, each equipped with a telegraph. In one room, the machine candidate for AI is connected to the telegraph, ready to receive and transmit messages. The second room, which is disconnected from the first room, houses a human confederate. A human judge resides in the third room with telegraph connections to the first and second room. The duty of the judge is to determine which room contains the human. Different instantiations of the Test vary details beyond this framework, e.g. how long the conversations go on before a judgment is made \cite{loebner03}. In all cases, the machine is deemed intelligent if it is able to trick the human judge into guessing that it is human.

In applying the principles of a test for intelligence outlined in 1.2, the limitations of the Turing Test come to the fore. The Test is not accessible in practical terms, since it relies on two humans, a judge and a confederate, per machine per test. The Test is generalizable insofar as natural language is AI-complete. The Test is not continuous; it outputs a ``pass'' or ``fail'' depending on whether the participant is able to fool the judge. The Test is very dependent on persons, in particular the human judge and confederate, and does not incorporate any popular conception of intelligence. Finally, the Test is immune to gaming, since it is not possible to anticipate the behavior of the human judge. Thus the Turing Test satisfies only two of the five prescribed principles for a practical test of intelligence.  This analysis is summarized in Table 1. The Test has withstood the test of time for good reason. Its main insight ---  that intelligence cannot be evaluated without an intelligent evaluator --- must be realized by any test designer. Nonetheless, as several have argued \cite{hayes95, moor01, shieber15}, the Turing Test should be viewed as a thought experiment rather than a practical test for intelligence. 

\subsection{Robotics Contests}

Robotics seems like a natural domain for testing artificial intelligence. The prospect of robots that are able to behave and reason at human levels is a clear motivation for AI research. Moreover, a candidate for AI is likely to be more convincing if it is physically instantiated. Anderson, Baltes, and Cheng (2011) review existing robotics contests and critique their utilities as benchmarks for AI research. These contests include annual competitions like AAAI/IJCAI, which consists of a diverse suite of tasks for the robots to perform \cite{balch02}; RoboCup, which requires candidates to play an actual game of soccer, thus entering the realm of multi-agent strategizing \cite{kitano97}; and HuroCup, which might be considered an ``olympics for robots'', requiring contestants to compete in a wide range of sports and agility competitions \cite{baltes08}. Anderson, Baltes, and Cheng find specific shortcomings in each of these competitions as proxies for AI, but suggest that more broad and versatile robotics competitions could still be useful for testing AI.

I argue that a test for AI should be independent of robotics. One inherent problem with robotics as a medium to test AI is that there are a host of extremely challenging problems in the field that have little or nothing to do with intelligence (e.g. the difficult mechanical problems associated with walking \cite{todd13}). It may be that certain problems in robotics are sufficient to demonstrate AI, but those problems are often harder than AI itself. Another fundamental problem in robotics competitions that currently exist is that they consist of a fixed set of a tasks. Even if these tasks are broad, this dependency is a violation of two of my principles for a practical test for AI: Dependent on People/Independent of Persons, and Immunity to Gaming. Any fixed set of tasks is susceptible to be criticized by third parties as unrepresentative of AI. Moreover, a robot can be trained to accomplish this fixed set of tasks without possessing any unifying architecture for general AI. Thus, robotics should be seen not as a medium for testing AI, but rather as an application of AI once it has been reached. Too keen of a focus on robotics threatens to pull research away from the most direct path towards AI. 

\subsection{Other AI Competitions}

There are several other existing competitions and prizes for artificial intelligence. Some of these take the form of general awards for career achievement, such as the annual David E. Rumelhart prize \cite{rumelhart} and the IJCAI Award for Research Excellence \cite{ijcai}. These awards serve to encourage research in AI, but do not claim to test AI nor directly steer the field towards it. Other contests focus on specialized tasks. ImageNet requires contestants to recognize objects in a massive image dataset \cite{russakovsky15}. The Hutter Prize offers a benchmark for lossless text compression \cite{mahoney09}. Several contests focus on AI game playing, such as chess \cite{hayes76}, poker \cite{littman06}, and general game playing \cite{genesereth05}. Few of these competitions claim to test general AI, and none are successful in satisfying all five principles of a practical test for AI (see Table 1). The search for a practical test for AI continues. 

\section{The Luna Rating System}

\subsection{Intelligence as a Social Construct}

Psychology and artificial intelligence offer several competing definitions of intelligence. Legg and Hutter collect 70 distinct definitions from both fields, admitting that even this list is incomplete \cite{legg07}. A unifying definition would be highly convenient for the purpose of designing a test. Alas, any one definition is revealed to be incomplete or flawed as soon as it is written down. Why is a definition for intelligence, a property with obvious societal importance, so elusive? I argue that intelligence is not an immutable property found in nature, but rather a social construct that depends completely on the society that constructs it. If humans did not exist, intelligence would not exist. 

Given this premise, it is clear that any test for artificial intelligence that does not receive direct input from human judges will be inherently flawed. A true test must capture the socially constructed definition of intelligence. In an ideal scenario, a machine would be evaluated by virtually every member of society using whatever methods of evaluation  each individual finds appropriate. A practical test must work to the same end under realistic constraints. Here I introduce the Luna Rating System as such a practical test and propose that it is an authentic reflection of the socially constructed notion of intelligence.

\subsection{The Luna Rating System}

The Luna Rating System (LRS) is a never-ending tournament of a two-player game called the Luna Game. Players of the Luna Game may be human or machine. All players have two consistent goals: to accurately evaluate the intelligence of their opponents, and to prove their own intelligence. Players are driven to demonstrate intelligence because they want to achieve a high \textit{Smarts Rating}. The Smarts Rating is the central quantity of interest for the purpose of identifying artificial intelligence. If a machine achieves a Smarts Rating that is on par with the Smarts Ratings of humans, it will have passed the implicit intelligence test of LRS and convincingly demonstrated general AI.

The details of the Luna Game are covered in Chapter 2. For the purpose of introducing LRS, it suffices to assume that a Luna Game requires each of the two players to evaluate the other's intelligence. A player's Smarts Rating is a reflection of the extent to which that player has convinced her past Luna Game opponents that she is intelligent. At the end of a Luna Game, a player's Smarts Rating is updated based on the evaluation of the opponent via Kalman filtering \cite{julier97}. Thus every player in LRS is constantly a judge of intelligence and a candidate under evaluation by other players.

One immediate advantage of LRS is its strong incentive for judges to perform their duties well. No other testing system for AI with human judges has a built-in impetus for high quality judging. Moreover, as the details of the Luna Game make clear, judges are not perversely motivated to attempt to fool candidates for AI by putting forth excessively difficult tests; instead, they are rewarded for giving judgments that are close to the consensus of all players in the system, and therefore close to the socially constructed definition of intelligence. 

\subsection{Analysis of Principles}

LRS is the only practical test for intelligence to satisfy all five of the principles outlined in 1.2. The system exists online and invites humans and machines to play for free. Thus LRS is maximally accessible. Like the Turing Test, the Luna Game involves open-domain question answering, and therefore is AI-complete, i.e. generalizable to all problems in AI. Smarts Ratings are continuous quantities that are updated after every Luna Game, making progress immediately clear to all players. These ratings reflect the equilibrium consensus of all players in LRS on what it means to be intelligence, and is not biased towards any single player's notion. Finally, the system is immune to gaming, since all players are encouraged to devise their own original questions, which cannot be predicted by the other player. Table 1 summarizes how the satisfaction of these five principals represents a substantial improvement over existing tests for AI.

\begin{table}[h!]
\centering
\label{my-label}
\begin{tabular}{|l|l|l|l|l|l|l|}
\hline
\textbf{Principle}                                                                              & \textbf{Turing} & \textbf{ImageNet} & \textbf{HuroCup} & \textbf{Hutter} & \textbf{Games} & \textbf{LRS} \\ \hline
Accessibility                                                                          &        & X        &         & X      & X     & X   \\ \hline
Generalizibility                                                                       & X      &          & X       &        &       & X   \\ \hline
Continuity                                                                             &        & X        & X       & X      & X     & X   \\ \hline
\begin{tabular}[c]{@{}l@{}}Dependent on People, \\ Independent of Persons\end{tabular} &        &          &         &        &       & X   \\ \hline
Immunity to Gaming                                                                     & X      &          &         &        & X     & X   \\ \hline
\end{tabular}
\caption{\textbf{The Luna Rating System is the only test that satisfies all five principles for a practical test of intelligence.} Other existing tests include the Turing Test, which invokes natural language; ImageNet, which focuses on object recognition in computer vision; HuroCup, a sports-based competition in robotics; the Hutter Prize, which deals with text compression; and several games, such as chess, poker, and general game playing.}
\end{table}

\subsection{Thesis Outline}

The primary contribution of this thesis is the introduction of the Luna Rating System as a practical test for machine intelligence. The remainder of the thesis is divided into three parts. In the next chapter, I continue the introduction of the Luna Rating System through a description of the Luna Game. This chapter is followed by a study of the robustness of LRS, characterizing likely strategies for game play and analyzing their effect on the accuracy of Smarts Ratings as a proxy for intelligence. In Part II, I enter LRS as a machine player. The Luna Game is broken down into three separate problems in natural language processing and machine learning. Each is given theoretical treatment and several novel solutions are proposed, many of which generalize beyond the scope of LRS. Finally, in Part III, I create the first online instantiation of LRS and invite humans and machines to play. Their games illuminate the current state of AI and provide a rich natural language dataset for further analysis.

%\pagebreak
%\bibliographystyle{abbrvnat}
%
%\begin{thebibliography}{}
%
%\bibitem{anderson11}
%Anderson, John, Jacky Baltes, and Chi Tai Cheng. ``Robotics competitions as benchmarks for AI research.'' The Knowledge Engineering Review 26.01 (2011): 11-17.
%
%\bibitem{anonymous14}
%Anonymous. ``Computer Simulating 13-year-old Boy Becomes First to Pass Turing Test.'' Http://www.theguardian.com/. The Guardian, 9 June 2014. Web.
%
%\bibitem{balch02}
%Balch, T. \& Yanco, H. 2002. ``Ten years of the AAAI mobile robot competition and exhibition.'' AI Magazine 23(1), 13?22.
%
%\bibitem{baltes08}
%Baltes, Jacky, and Thomas Braunl. ``HUROCUP: General Laws of the Game 2008.'' (2009).
%
%\bibitem{genesereth05}
%Genesereth, Michael, Nathaniel Love, and Barney Pell. ``General game playing: Overview of the AAAI competition.'' AI magazine 26.2 (2005): 62.
%
%\bibitem{hayes76}
%Hayes, Jean E., and David NL Levy. The world computer chess championship, Stockholm 1974. Edinburgh University Press, 1976.
%
%\bibitem{hayes95}
%Hayes, Patrick, and Kenneth Ford. ``Turing test considered harmful.'' IJCAI (1). 1995.
%
%\bibitem{ijcai}
%``IJCAI Award for Research Excellence.'' International Joint Conferences on Artificial Intelligence. N.p., n.d. Web. 12 Dec. 2015.
%
%\bibitem{julier97}
%Julier, Simon J., and Jeffrey K. Uhlmann. ``New extension of the Kalman filter to nonlinear systems.'' AeroSense'97. International Society for Optics and Photonics, 1997.
%
%\bibitem{kitano97}
%Kitano, Hiroaki, et al. "Robocup: The robot world cup initiative." Proceedings of the first international conference on Autonomous agents. ACM, 1997.
%
%\bibitem{lecun98}
%LeCun, Yann, Corinna Cortes, and Christopher JC Burges. ``The MNIST database of handwritten digits.'' (1998).
%
%\bibitem{legg07}
%Legg, Shane, and Marcus Hutter. ``A collection of definitions of intelligence.'' Frontiers in Artificial Intelligence and applications 157 (2007): 17.
%
%\bibitem{littman06}
%Littman, Michael, and Martin Zinkevich. ``The 2006 AAAI computer poker competition.'' ICGA Journal 29.3 (2006): 166.
%
%\bibitem{loebner03}
%Loebner, Hugh. ``Home page of the Loebner prize-the first Turing test.'' Online unter http://www. loebner. net/Prizef/loebner-prize. html (2003).
%
%\bibitem{mahoney09}
%Mahoney, Matt. ``Rationale for a Large Text Compression Benchmark.'' Rationale for a Large Text Compression Benchmark. N.p., 23 July 2009. Web. 12 Dec. 2015.
%
%\bibitem{marcus14}
%Marcus, Gary. ``What Comes After the Turing Test?'' The New Yorker. The New Yorker, 09 June 2014. Web. 12 Dec. 2015.
%
%\bibitem{moor01}
%Moor, James H. ``The status and future of the Turing test.'' Minds and Machines 11.1 (2001): 77-93.
%
%\bibitem{rumelhart}
%``For Contributions to the Theoretical Foundations of Human Cognition.'' The David E Rumelhart Prize RSS. N.p., n.d. Web. 12 Dec. 2015.
%
%\bibitem{russakovsky15} 
%Olga Russakovsky*, Jia Deng*, Hao Su, Jonathan Krause, Sanjeev Satheesh, Sean Ma, Zhiheng Huang, Andrej Karpathy, Aditya Khosla, Michael Bernstein, Alexander C. Berg and Li Fei-Fei. (* = equal contribution) ImageNet Large Scale Visual Recognition Challenge. IJCV, 2015.
%
%\bibitem{shieber14}
%Shieber, Stuart M. No, the Turing Test has not been passed. In The Occasional Pamphlet. 10 June 2014a. URL http://blogs.law.harvard.edu/pamphlet/2014/06/ 10/no-the-turing-test-has-not-been-passed/.
%
%\bibitem{shieber15}
%Shieber, Stuart M. ``Principles for Designing an AI Competition, or Why the Turing Test Fails as an Inducement Prize'' Forthcoming. (2015).
%
%\bibitem{todd13}
%Todd, David J. Walking machines: an introduction to legged robots. Springer Science \& Business Media, 2013.
%
%\bibitem{turing50}
%Turing, Alan M. ``Computing machinery and intelligence.'' Mind (1950): 433-460.
%
%\bibitem{you15}
%You, Jia. ``Beyond the Turing Test.'' Science 347.6218 (2015): 116-116.
%
%%\end{thebibliography}
%
%\end{document}
%\begin{savequote}[75mm]
But it is not conceivable that such a machine should produce different arrangements of words so as to give an appropriately meaningful answer to whatever is said in its presence, as the dullest of men can do.
\qauthor{Ren\'e Descartes}
\end{savequote}

\chapter{The Luna Game}

\section{Overview}

At the center of my proposed rating system is a two-player game that I call the Luna Game. Each player enters the game with a Smarts Rating, which has been assigned based on her performance in previous games. As the name suggests, the Smarts Rating is a proxy for the player's intelligence. The Smarts Rating of each player is hidden from the other player until the end of the game. The objective of the Luna Game is simple: guess the Smarts Rating of the other player. In other words, a player should strive to accurately evaluate the intelligence of her opponent. The winner of the Luna Game is the player whose guess is closest to the actual Smarts Rating of the other player. After the game, the opponent's guess is factored into the player's Smarts Rating so that the rating captures all the guesses of previous opponents.

As a player with a high Smarts Rating, why not ``play dumb''? This strategy would indeed induce an inaccurately low guess from the opponent, possibly leading to a win. However, the motives of a player reach beyond the scope of a single game. In addition to winning games, a player wants to achieve a high Smarts Rating. Since the rating depends on the guesses of all the player's opponents, she will need to ``play smart'' to accomplish her long term goal. The ``playing dumb'' method is not only detrimental to a player's rating, but also unsustainable as a consistent strategy; a player of that method will have her Smarts Rating lowered as a result, narrowing the margin between future opponents' guesses and her actual Smarts Rating if she continues to use the strategy. Players who remain and thrive will be those who play smart.

In designing the Luna Game, I sought to impose as few constraints as possible. The Game is meant to be a microcosm of the organic process for defining intelligence. Humans evaluate each other's intelligences through a series of questions and answers, often in the form of a written exam, but also informally through everyday conversations. The most natural notion of an individual's intelligence arises from the consensus of the people who perform these evaluations. A player's Smarts Rating is meant to reflect this natural notion; it is an aggregate of evaluations carried out by other players. To define the scope of an evaluation, I impose only those constraints necessary to motivate honest and repeated play.

A session of the Luna Game consists of three phases: the Interview Phase, the Response Phase, and the Guess Phase. During the Interview Phase, each player creates a set of five questions to pose to the opponent. In the Response Phase, each player responds to the other's questions. Finally, in the Guess Phase, each player receives responses back from her opponent, and must use the responses to guess her opponent's Smarts Rating. I describe each of these phases in detail throughout the rest of this chapter and illustrate the game through examples of play.

\section{Interview Phase}

A Luna Game begins with the Interview Phase. During this phase, each player prepares a set of five free-form questions to be given to the other player. The number of questions represents a tradeoff between the time required to complete the phase and the difficulty of the guessing task. With more questions, each player would need more time to construct the questions, increasing the likelihood that they will quit the game and leave the system. With fewer questions, construction time could be shortened, but the informativeness of the subsequent responses would suffer, and the Smarts Ratings would ultimately be less meaningful. I chose the number five to optimize this tradeoff, but the number may be adjusted in future iterations of the system. In a similar practical vein, I insist that questions be constructed in batch, rather than allowing for sequential question-response. Since the game is played online, allowing for back and forth would increase the length of each game and significantly decrease the probability that a game gets finished.

The other major consideration in the design of the Interview Phase is the form of the questions. The only constraint I impose is a limit of $5000$ characters per question. I do not insist that questions be actual questions, nor that they be in any particular language, nor that they expect a particular form of response. In natural language terms, the questions are of open domain, since I do not restrict the content of questions. I recognize that in practice, players may opt for yes-or-no or multiple choice questions, which could simplify the task of learning to guess ratings. Players may also limit the domain of their questions, since general form questions may be too difficult for current AI, so general questions may not meaningfully differentiate between them. Nonetheless, I leave the choice of question topic and form to the players themselves. I anticipate that players will find the optimal question types better than I as the game designer could, and that the question types will naturally evolve in correspondence with the evolution of the AI players.

\subsection{Instructions}

The following instructions are presented during the Interview Phase.
\begin{center}
\textit{You are now in the Interview Phase. Please enter a list of five questions for the other player. Keep in mind the following strategic hints:}

\begin{itemize}
\item \textit{Your questions should be as informative as possible for guessing the other player's Smarts Rating.}
\item \textit{Your questions should have a very wide range of difficulties.}
\item \textit{Your questions need not have ``right'' or ``wrong'' answers.}
\item \textit{Search engine access is allowed, so trivia questions will not be very informative.}
\item \textit{Do not assume that the other player is human!}
\end{itemize}
\end{center}

\subsection{Interview Strategy}

In preparing questions, a player knows nothing about her opponent. She must prepare for extremes --- a completely naive machine opponent or a very clever human opponent  --- and she also must be able to differentiate between players with Smarts Ratings in the middle of the spectrum. Given the competitive nature of the game, a player may be tempted to create a set of extremely hard questions. This choice would prove unwise, since the player will be unable to accurately guess the Smarts Rating of an opponent who gets all of the questions ``wrong''. A question set that is too easy will lead to the opposite problem. Thus an ideal set of questions will have a wide range of difficulty.

\subsection{Examples}

Below is an example of a question set. I choose questions from the web-based implementation of LRS described in Chapter $6$ to illustrate the range of possible question types and to demonstrate appropriate levels of difficulty. Early questions are aimed at differentiating between naive machines, while later questions are directed towards advanced human players. Each question is designed to induce a response that will reveal the intelligence of the opponent.

\begin{enumerate}
\item Do you like games?
\item People who live in Boston are called Bostonians. What is a person who lives in Cambridge, MA called?
\item l -|- l = ?
\item How do you define success?
\item If a hacker can determine when keys on your keyboard are pressed (without knowing which keys), how are you in danger?
\end{enumerate}

\section{Response Phase}

The Response Phase is a player's opportunity to convince her opponent that she is intelligent. Questions are received as soon as both players have finished the Interview Phase. Each player must then respond in free form to all five questions. Responses are not returned until both players have finished answering all questions. Like questions in the Interview Phase, responses are unconstrained in form, and only limited in length to $5000$ characters each. A player is motivated by the prospect of an increase in Smarts Rating to respond to the questions thoroughly and to the best of her ability.

\subsection{Instructions}

The following instructions are presented during the Response Phase.
\begin{center}
\textit{The other player has sent you questions! You are now in the Response Phase of the Luna Game. Please answer the following questions: [Question Set]. In answering the questions, keep in mind the following strategic hints:}
\begin{itemize}
\item \textit{You should answer the questions to the best of your ability.}
\item \textit{The other player will use your answers to guess your Smarts Rating.}
\item \textit{The higher the other player guesses, the higher your Smarts Rating will become.}
\end{itemize}
\end{center}

\subsection{Examples}

The responses below are also taken from the web-based implementation of LRS described in Chapter $6$.

\begin{enumerate}
\item Q: Do you like games? \\
A: Yes I love games
\item Q: People who live in Boston are called Bostonians. What is a person who lives in Cambridge, MA called?\\
A: An academic
\item Q: l -|- l = ?\\
A: Why are you using capital I's, and what in the world is ``-|-''?
\item Q: How do you define success?\\
A: Dictionary.com defines it as ``the favorable or prosperous termination of attempts or endeavors; the accomplishment of one's goals.''
\item Q: If a hacker can determine when keys on your keyboard are pressed (without knowing which keys), how are you in danger?\\
A: Ugh, this is a difficult one. It would make guessing password easier maybe, because the hacker would know the length of a password. It also depends on what other info is available to the hacker, such as Web addresses or sites visited. Hacker could also known and record when (times each day) the computer is not in use, making it easier to remotely control the computer without the user knowing.
\end{enumerate}

\section{Guess Phase}

After both players have responded to each other's questions, their responses are returned for evaluation. Each player then must formulate a guess of the other's Smarts Rating based on these responses. In practice, the player might also attempt to take into account the questions provided by the opponent, but since questions may be generated automatically, it is advisable to focus on the opponent's responses. The winner of the Luna Game is the player whose guess is closest to the actual Smarts Rating of her opponent.

A competitive player may consider guessing the lowest possible rating, knowing that the game will be lost, but the opponent's Smarts Rating will decrease as a result of the guess. However, this strategy offers no real benefit to the player, since Smarts Ratings are not rankings; the player's Smarts Rating will not improve as a result of the opponent's Smarts Rating suffering. (Nonetheless, I analyze the system-level effects of this strategy in Chapter $3$.) Thus the only rational strategy for guessing is to attempt to guess as close as possible to the actual Smarts Rating of the opponent.

\subsection{Instructions}

The following instructions are presented during the Guess Phase.
\begin{center}
\textit{The other player has responded to your questions! You are now in the Guess Phase of the Luna Game, which is the final phase. Below are the other's answers: [Answer Set] Based on these answers, please enter a guess of the other player's Smarts Rating.}
\end{center}
In addition, I provide functionality that encourages the player to evaluate each question individually on a sliding scale from 0 to 100. I prompt the player to assign the single question score by asking, ``How smart was this response?'' This process is optional, as I do not want to slow down the impatient player. However, a player is incentivized to use the sliding scales if they do not have a more sophisticated method of guessing ratings, since the single question scores can be automatically converted into a guess. These single question scores provide insight into the hardness of the natural language questions, effectively creating a dataset of questions annotated with difficulty.

\section{Game Conclusion}

After both players have provided guesses, the Luna Game is complete. The winner of the game is the player whose guess is closest (in terms of $L_1$ distance) to the actual Smarts Rating of their opponent. It is possible, though unlikely, for the game to end in a tie if the distance between guess and actual is equal for both players. In addition to reporting the outcome of the game, the system reveals the actual Smarts Rating of the opponent and the opponent's guess. The system also updates the players' Smarts Ratings so that it is the mean of all previous human opponent Guesses and reports these updates to the respective players. The mean was chosen for simplicity, though more sophisticated statistics that are adaptive, such as Elo Ratings, could also be used in the future. Machine guesses are not factored into Smarts Ratings.

\subsection{Example}
Below is an example of feedback at the end of a Luna Game.
\begin{center}
\textit{Your Luna Game is complete! Below are the results.}\\
\textit{Game Outcome: You won!}\\
\textit{Your New Smarts Rating: $78$}\\
\textit{Actual Other Player Smart Rating: $84$ (You guessed $81$)}\\
\textit{Other Player's Guess of Your Rating: $91$ (Your rating was $75$)}
\end{center}

\subsection{Conclusion of a Player's First Luna Game}

New players do not have Smarts Ratings until the end of their first game, at which point they are assigned the guess of their first opponent. The game is counted as an automatic win for the opponent, but not as a loss for the new player. This process is to avoid the possibility of the Smarts Rating equilibrium collapsing into a constant (see Chapter $3$).
%\begin{savequote}[75mm]
It might be urged that when playing the ``imitation game'' the best strategy for the machine may possibly be something other than imitation of the behaviour of a man. This may be, but I think it is unlikely that there is any great effect of this kind.
\qauthor{Alan Turing}
\end{savequote}

\chapter{Robustness of the Luna Rating System}

\section{The Smarts Rating as a Proxy for Intelligence}

The Luna Rating System is only effective as a test if Smarts Ratings genuinely reflect intelligence. Without appropriate safeguards, Smarts Ratings can quickly lose their integrity. To illustrate this potential danger, consider a version of LRS that initializes all new players with a Smarts Rating of $50$. In the early days of LRS, if this initialization is known to all players, the strategy of always guessing $50$ will do quite well. In fact, if all players guess rationally, no rating will ever deviate from $50$, and every Luna Game will end in a tie. As more players join the system, if the equilibrium has already been fixed at this constant, no rational force will make it budge. This outcome would render the version of LRS unusable for testing intelligence and uninteresting for players. Clearly LRS must be implemented with care.

This chapter studies the extent to which a player's Smarts Rating may be different from the player's intelligence. I formalize intelligence to be the expected ``Actual Guess'' made by an opponent of the player's Smarts Rating. Trouble arrives when players choose to give a ``Reported Guess'' that is different from an Actual Guess. From this discrepancy emerges distance between the Smarts Rating and intelligence, i.e. error in Smarts Ratings. It is theoretically possible for Smarts Ratings to be arbitrarily far from intelligence; if all players report random or adversarial Guesses, Smarts Ratings will be meaningless. However, it is safe to assume that most players seek Luna Game wins, high Smarts Ratings, or both. With these motivations, there are several likely strategies that players may choose among. The robustness of LRS can be assessed through the analysis of these strategies and their cumulative effects on Smarts Ratings.

\subsection{Demonstrated Intelligence}

In considering the validity of Smarts Ratings, a natural question is whether the intelligence demonstrated by players during the Luna Game is indicative of ``actual'' intelligence. Assuming that players are capable of demonstrating intelligence through language in real life, it is clear that players may demonstrate intelligence during a Luna Game. But what if a player chooses not to demonstrate intelligence? What if a player is capable of answering with high intelligence, but decides to provide a subpar answer, perhaps out of laziness or misguided strategy? In this case, the Smarts Rating of the player may not reflect their capacity for intelligence. However, from the perspective of LRS as a test for intelligence, this possibility is not at all problematic. LRS is ultimately a test for \textit{demonstrated intelligence}. It evaluates the intelligence of responses, not the intelligence of respondents. Indeed, any test for intelligence is inherently an evaluation of demonstrated intelligence; it is impossible to assess anything else. 

\subsection{Reported and Actual Guesses}

If a player withholds a strong response in favor of a weak response, the withheld response is irrelevant for LRS. However, if a player withholds an honest Guess of the opponent's Smarts Rating in favor of a dishonest one, this can have negative consequences for the validity of LRS. If a student fails an exam, that does not mean the exam has failed its purpose, but if an exam is graded incorrectly, its value is lost. Thus in studying the dynamics of Smarts Ratings, a distinction must be drawn between Reported Guesses and Actual Guesses. The system must be designed in such a way that Reported Guesses are equal to Actual Guesses as often as possible. I explore the consequences of misalignment below.


\subsection{Definitions and Notation}

Let $P$ be the set of all players in the LRS. Let $\mathcal{P}$ be a random variable that assumes values over $P$ with uniform probability. Let $SR \subseteq \mathbb{R}^+$ be the domain of Smarts Ratings. The definitions of Reported Guess and Smarts Rating are mutually dependent.

\theoremstyle{definition}
\begin{definition}{Reported Guess}\\
For players $p, p' \in P$, player $p$ has a \textit{Reported Guess} of the Smarts Rating of player $p'$, written as $G(p, p') \in SR$.
\end{definition}

\noindent The definition of Reported Guess assumes that players do not make decisions stochastically, nor based on their state. It is possible to weaken this assumption and arrive at similar conclusions.

\theoremstyle{definition}
\begin{definition}{Smarts Rating}\\
The \textit{Smarts Rating} of a player $p \in P$ is the expected Reported Guess of an opponent. This is written as $S(p) \triangleq E[G(\mathcal{P}, p)]$.
\end{definition}

\noindent The given definition of Smarts Ratings is an expectation rather than a quantity that depends on the number of Games played. This definition is expedient for theoretically evaluating the discrepancy between Smarts Ratings and intelligence. Later I provide simulations that probe the role of time in this discrepancy. Additionally, note that this definition does not preclude a player from playing herself, which is of course impossible in practice. Incorporating this observation into the definition would have no meaningful effect on results and would only muddle the analysis, hence the exclusion.

\theoremstyle{definition}
\begin{definition}{Actual Guess}\\
For players $p, p' \in P$, player $p$ has an \textit{Actual Guess} of the Smarts Rating of player $p'$, written as $G^*(p, p') \in SR$.
\end{definition}

\noindent The Actual Guess may or may not be different from the Reported Guess.

\theoremstyle{definition}
\begin{definition}{Intelligence}\\
The \textit{intelligence} of a player $p \in P$ is the expected Actual Guess of an opponent. This is written as $I(p) \triangleq E[G^*(\mathcal{P}, p)]$.
\end{definition}

\section{Theoretical Effects of Strategies on Smarts Rating Validity}

LRS presents two goals for players: to win Luna Games, and to achieve high Smarts Ratings. The spirit of the game encourages players to answer all questions as they would in a real world context, and to guess the Smarts Rating of an opponent in the same manner as they would evaluate the intelligence of a human. However, players may ignore the spirit of the game and adopt strategies that they think will maximize payoff in terms of both goals. They may also decide to focus on only one of the two goals, strategizing to maximize Luna Game wins at the expense of Smarts Ratings, or vice versa. If the Smarts Rating is to be used as a proxy for intelligence, the equivalence between the two must be robust to any likely player strategies.

\subsection{Honest Play}

If players always have Reported Guesses that are the same as their Actual Guesses, then Smarts Ratings will align with intelligences over time. This follows directly from the formalized definition of intelligence given above. Thus LRS designers should take every measure to encourage honest play.

\subsection{Single Priority Play}

One must always be prepared for players to ignore the spirit of the game and break any rules that aren't technically enforced. Honest play assumes that players give their best guess for any player's Smarts Rating. It also assumes that players present themselves as intelligently as possible with the aspiration of winning individual Luna Games. But what if a player cares only about Smarts Ratings and is indifferent towards the outcome of Luna Games? Or what if the opposite occurs, with a player ignoring Smarts Ratings and focusing only on winning Luna Games? In this section, I explore the system-level effects in both of these cases.

\subsubsection{Luna Game Win Maximization}

Suppose that a player is indifferent to her Smarts Rating and cares only to maximize her expected number of Luna Game wins. With the incentive of winning a game, it is clear that the player will provide a Reported Guess that is equal to her Actual Guess. The only aspect of the game that this player may manipulate is her responses. For example, if she currently has a very high Smarts Rating, she may provide responses that are indicative of a very low Smarts Rating, inducing a low Guess from the opponent and increasing the probability of winning. The optimal strategy may be to oscillate between maximally intelligent responses and minimally intelligent ones, maintaining a Smarts Rating near the center of the range of possible ratings. 

In any case, strategizing by manipulating responses does not in any way jeopardize the validity of Smarts Ratings. As described above, LRS does not claim to be a test for capacity for intelligence; it can only be a test of demonstrated intelligence. If a human exhibits highly intelligent behavior one day and minimally intelligent behavior the next day, their overall intelligence is judged to be somewhere in between. As an average of play over time, the Smarts Rating captures this intuitive notion of intelligence better than any one-time evaluation could. Thus players may be able to ``game their opponents'' by oscillating the quality of their responses, but they will not be able to game the system.

\subsubsection{Smarts Rating Maximization}

Suppose that a player ignores Luna Game outcomes and strategizes to maximize her Smarts Rating at all costs. The best response strategy is clearly to respond as intelligently as possible at all times, which is in line with the spirit of LRS. However, if the player interprets Smarts Ratings as relative measures, the optimal guessing strategy is to give a Reported Guess of $0$ regardless of the opponent's play. This strategy is optimal because the player will perceive a slight benefit from a decrease in the opponent's rating. 

I assess the net impact of this Minimum Guessing strategy on LRS validity from two angles. First, to evaluate the likelihood that the strategy is adopted, I quantify the expected benefit of adopting this strategy for an individual player. I show that as a function of the number of players in the system, this benefit is so small that a player concerned only marginally with winning Luna Games will be better off playing honestly. Second, in the event that a player does adopt this strategy, I quantify the error that Minimum Guessing introduces to Smarts Ratings as a function of the number of players that adopt it.

Consider the Minimum Guessing strategy from the perspective of an individual player. If the player truly attaches zero value to a Luna Game win, then always guessing $0$ makes sense. But if the player cares even a marginal amount about wins, the personal benefit from guessing $0$ is unlikely to outweigh the cost. Let $p \in P$ be a player deciding between honesty and Minimum Guessing. Note that the Smarts Rating of $S(p)$ will be unchanged regardless, since $p$'s strategy choice will not affect the Guessing strategies of opponents. Let $s_1$ the mean Smarts Rating if $p$ Guesses honestly, where $N$ is the number of players in the system. We can express $s_1$ as
\begin{center}
\begin{math}
s_1 = \frac{1}{N}(\sum_{p' \in P}S(p')) = \frac{1}{N^2}(\sum_{p' \in P}(\sum_{p'' \in P}G(p'', p')))
\end{math}
\end{center}

\noindent Let $s_2$ be the mean Smarts Rating if $p$ instead uses the Minimum Guessing strategy. In this case, each Guess $G(p, p')$ in the expression of $s_1$ will change to $0$. Therefore the new mean Smarts Rating in this scenario will be 
\begin{center}
\begin{math}
s_2 = s_1 - \frac{1}{N^2}(\sum_{p' \in P} G(p, p'))
\end{math}
\end{center}
The relative benefit to the $p$ in choosing the Minimum Guessing strategy is thus \\
$\frac{1}{N^2}(\sum_{p' \in P} G(p, p'))$. For perspective, suppose that $p$ has Actual Guesses that are perfect, i.e. they match the Smarts Ratings of other players. Then the relative benefit is $\frac{1}{N}$ of the average Smarts Ratings of all players in the system.  As the number of players in the LRS increases, this change quickly becomes negligible. Thus any rational player with even the slightest desire to avoid losing every Luna Game will likely abstain from the Minimum Guessing strategy.

While the Minimum Guessing strategy is evidently subpar for players caring at all about winning, there may still be players who choose to adopt it. Suppose that there are $K$ such players, indexed $p_1, p_2, ..., p_K \in P$, with all other players adhering to Honest Guessing. I quantify the expected error introduced to the Smarts Rating of another player $p \in P$ as a result of these $K$ rogue players. Let $p \in P$. The most accurate Smarts Rating, i.e. the intelligence, is given by
\begin{center}
\begin{math}
I(p) = \frac{1}{N}(\sum_{p' \in P}(G^*(p', p)))
\end{math}
\end{center}
The Smarts Rating actually assigned to $p$ in this scenario is
\begin{center}
\begin{math}
S(p) = \frac{1}{N}(\sum_{p' \in P}(G(p', p))) = \frac{1}{N}((\sum_{p' \in P}(G^*(p', p)) - (\sum_{i=1}^K(G^*(p_i, p))))
\end{math}
\end{center}
Thus the total error introduced by these $K$ players is $\frac{1}{N}(\sum_{i=1}^K(G^*(p_i, p)))$. At a high level, this equation tells us that if half the players in the system use Minimum Guessing, then a player's Smarts Rating will be roughly half the player's intelligence. This impact suggests that LRS is tolerant to a small minority of Minimum Guessing players, but measures should be taken to discourage and prevent wide use of the strategy.

\subsection{Response Agnostic Play}

How should a first-time Luna Game player formulate her Guess? There are several possible sources of information that she may utilize. She has her opponent's responses, and she may be able to estimate the intelligence of the opponent's responses based on experiences outside LRS. However, the responses and intelligence estimate alone are not enough to provide a Guess; she also needs to understand the meaning and scale of a Guess. Here LRS instructions may provide three sources of insight: the initialization value of Smarts Ratings, the domain of Smarts Ratings, and the distribution of Smarts Ratings for players currently in the system. In the extreme case, a player may ignore opponent responses completely and utilize only the statistics provided by LRS. How would such strategizing affect the validity of Smarts Ratings? This question is of import not only for first-time players, but also for experienced players who might incorporate these statistics into their overall strategies if they are available.

\subsubsection{Known Initialization}

As described in the introduction of this chapter, a known constant initialization value for Smarts Ratings could lead to an abrupt collapse in Smarts Ratings. Many early players would likely recognize that ($1$) new players have the same Smarts Rating and ($2$) most players are new, and therefore ($3$) the strategy of guessing the initialization value is highly effective. The result would be a quick collapse in Smarts Ratings to the initialization value, from which the system could not recover. 

There are three alternatives to initialization with a constant value. The first is initialization according to some predefined ``quiz''. This option is unattractive, not only because it primes players to think of intelligence in a particular way, but also because it induces players to ask similar questions to try to mimic the quiz. The second alternative is random initialization. While avoiding theoretical problems, this option has the practical downside of discouraging players who are randomly initialized low Smarts Ratings. It also makes the opponent's job of Guessing first time players an arbitrary endeavor.

The third and preferable alternative to fixed initialization is to forgo initialization altogether. A player's Smarts Rating can be assigned \textit{after} her first Luna Game, and it will be equal to the Guess of her opponent. With this configuration, there is no initialization that makes sense for players to guess as a default. The only downside of this approach is that the outcome of the first Luna Game loses meaning. To avoid discouraging experienced players, a Luna Game between a first time player and a non-first time player will result in a win for the latter, but will not be counted as a loss for the first time player. A Luna Game between two first time players will result in a tie. This third option is the one used in the implementation of LRS described in this thesis.

\subsubsection{Known Distribution}

A seemingly natural feature to include in an instantiation of LRS is the ability to see the distribution of all Smarts Ratings in the system. Revealing the distribution makes it possible for players to adopt Guessing strategies that take advantage of this distribution. This is not inherently bad for LRS; for example, the strategy of randomly sampling from the distribution for Guessing will preserve the distribution. However, other strategies involving the distribution can detrimentally affect the distribution. For example, suppose that a player's instinctive Actual Guess is the highest value presently in the distribution. The player may reason that the probability of her current opponent actually being the best player in the system is low, and therefore provide a Guess that is lower than the maximum, even if the opponent actually is the best player. The net effect will be a collapsing towards the center of the distribution, as witnessed in the case of a fixed known initialization value.

The most extreme strategy that takes advantage of a known distribution is to always Guess the mean Smarts Rating in the current distribution. It is clear that if all players adopted this strategy, all Smarts Ratings would converge to a constant. More realistic is the supposition that some players will judge opponents using percentiles, e.g. ``my opponent is smarter than $75$\% of other players'', and then map this judgment to the percentiles evident in the Smarts Ratings distribution, e.g. ``$75$\% of players have Smarts Ratings less than or equal to $55$, so my Guess is $55$.'' This strategy uses the Quantile Function of the Smarts Rating distribution, and thus it is referred to as Quantile Guessing.

Let $Q : [0, 1] \to SR$ be the Quantile Function for the Smarts Rating distribution, and let $F_p : P \to [0, 1]$ be the ``player percentile function'' that $p$ uses to map other players to percentiles according. This function may be thought of as a cumulative distribution function over $\mathcal{G}_p = \{G(p, p') : p' \in P\}$, i.e. $F_p(p') = P[X \le G(p, p')]$ where $X \sim \mathcal{G}_p$. If player $p$ uses Quantile Guessing, her Reported Guess of player $p' \in P$ will be $G(p, p') = Q(F_p(p'))$. If Quantile Guessing is used by all players from the beginning of LRS, Smarts Ratings will again collapse to a constant. Early players will know that there are only a small number of possible values for Smarts Ratings and will guess by selecting one, which will quickly lead to the ratings converging to one value. It is less clear if Quantile Guessing is problematic once a wide distribution of Smarts Ratings has already been established by players of other strategies. For insight here, I turn to simulations.

\section{Simulations}

Thus far I have outlined the major Luna Game strategies and described their theoretical implications on the validity and volatility of Smarts Ratings. I next provide simulation results to further stress test LRS against various strategies. Simulations are necessary not only to corroborate the theory, but also to ask questions that cannot be cleanly answered otherwise. For example, whereas the strategies have only been discussed individually so far, simulations allow for a comprehensive study of combined strategies. The theoretical section also did not explore the amount of time required to reach each equilibrium; such questions are left to simulations. The resulting collection of theoretical and simulation insights lays a foundation for robust LRS design.

\subsection{Methods}

The recruitment of human players is pivotal to the success of LRS because it is not otherwise clear what form Actual Guess functions will take. For the purpose of simulations, I create $N$ simulated players and assign each a uniformly random value between $0$ and $100$. The Actual Guess function for a given player is created by adding random Gaussian noise $(\mu = 0, \sigma^2 = 5)$ to each of the initialized values. Thus a player's intelligence, defined as the expected Actual Guess, will be very close to the initialized random value for that player. The Reported Guess for each player varies according to the experiments, following one of the strategies described in ($2$). 

An experiment runs for $T$ time steps. At each time step, two players are randomly selected from the pool of all $N$ players. Each player reports her Reported Guess of the other's Smarts Rating, and each player's Smarts Rating is updated so that it is the mean of all previous Reported Guesses. After the $T$ time steps, the main dependent variable of interest is the $L_1$ error between players' Smarts Ratings and intelligences. I measure both the average error and the maximum error. For some strategies, the independent variable of interest is $N$, and for others it is $T$. If $N$ is kept constant, it is done so at $N=100$; if $T$ is kept constant, it is done so at $T = 1000$. All configurations are run $100$ times and averaged over the trials.

\subsection{Results}

\subsubsection{Baseline}

Since Smarts Ratings and intelligences are both on a scale from $0$ to $100$, one baseline for the $L_1$ distance between the two is achieved by assuming that all players give the same constant Reported Guess, such as $0$ (which would happen if all players use Minimum Guessing). Doing so would result in an expected error of $50$. Another baseline is given by the expected distance between two uniformly randomly selected points from the domain of Smarts Ratings. For two random variables $A_1$ and $A_2$ drawn independently from the uniform distribution over $[0, 100]$, let $B$ be a random variable given by $B = max(A_1, A_2)$, and let $C$ be a random variable given by $C = min(A_1, A_2)$. By symmetry, $E[100 - C] = E[B]$. Furthermore, the expected value of $B$ is half that of $C$ (this can be derived by conditioning on the values of $A_1$ and $A_2$), so $2E[B] = E[C] \implies E[C-B] = E[B]$. Now note that $B + (C-B) + (100-C) = 100$, so $3E[C-B] = 100 \implies E[C-B] = \frac{100}{3}$. Thus $33.33$ is one appropriate baseline for the $L_1$ error of Smarts Ratings.

A third baseline can be derived by assuming that all players give fixed random Reported Guesses (independent from Actual Guesses). In this case, all Smarts Ratings will approach the expected value, and the error will be the expected difference from the expected value to a randomly drawn Smarts Rating. In terms of the parameters used in this simulation, Figure \ref{fig:RandomGuess} shows that mean error starts around the $33.33$ baseline and approaches $25$, while maximum error stays above $50$. 


\begin{figure}[h]
\centerline{%
\includegraphics[width=0.5\textwidth]{figures/robustness/Random_Guessing21.png}%
\includegraphics[width=0.5\textwidth] {figures/robustness/Random_Guessing22.png}%
}%
\caption{Mean and maximum $L_1$ error of Smarts Ratings when all $N$ players use a Random Guessing strategy. In Random Guessing, a player's Reported Guess is a random element from the domain of Smarts Ratings that is completely independent from the Actual Guess. These results establish baselines for subsequent simulations.}
\label{fig:RandomGuess}
\end{figure}

\subsubsection{Honest Play}

Honest players give Reported Guesses that are equivalent to their Actual Guesses. If all players are honest, Smarts Ratings quickly converge to Intelligences, as shown in Figure \ref{fig:HonestGuess}. Inducing honest play should be the foremost goal of LRS designers.

\begin{figure}[h]
\centerline{%
\includegraphics[width=0.5\textwidth]{figures/robustness/Honest_Guessing21.png}%
\includegraphics[width=0.5\textwidth] {figures/robustness/Honest_Guessing22.png}%
}%
\caption{Mean and maximum $L_1$ error of Smarts Ratings when all $N$ players Guess honestly. Honest play is defined by consistent equivalence between the player's Reported Guesses and Actual Guesses. Smarts Ratings quickly converge to intelligences if all players are honest.}
\label{fig:HonestGuess}
\end{figure}

\subsubsection{Single Priority Play}

If a player cares only about maximizing relative Smarts Ratings and neglects Luna Game outcomes, that player will likely give Reported Guesses of $0$ for all opponents. If all players use this Minimum Guessing strategy, Smarts Ratings will immediately collapse to $0$. More generally, the error introduced by Minimum Guessing depends linearly on the number of players using this strategy. Figure \ref{fig:MinimumGuess} confirms this result suggested by the theory. Mean $L_1$ error ranges from half the total range of Smarts Ratings to $0$.

\begin{figure}[h]
\centerline{%
\includegraphics[width=0.5\textwidth]{figures/robustness/Minimum_Guessing31.png}%
\includegraphics[width=0.5\textwidth] {figures/robustness/Minimum_Guessing32.png}%
}%
\caption{Mean and maximum $L_1$ error of Smarts Ratings when a fraction of the $N$ players use Minimum Guessing after $T$ time steps. In Minimum Guessing, a player always gives Reported Guesses of $0$. The error introduced by Minimum Guessing depends linearly on the number of players using the strategy.}
\label{fig:MinimumGuess}
\end{figure}

\subsubsection{Response Agnostic Play}

If the distribution of current Smarts Ratings is known at all times to all players, some players may try to take advantage of the distribution to formulate their Guesses. A naive approach is to give Reported Guesses equal to the current mean Smarts Rating. With all players using this approach, the distribution collapses and error explodes. In general, Figure \ref{fig:MeanGuess} shows that the error introduced by Mean Guessing depends linearly on the number of players using the strategy.

\begin{figure}[h]
\centerline{%
\includegraphics[width=0.5\textwidth]{figures/robustness/Mean_Guessing31.png}%
\includegraphics[width=0.5\textwidth] {figures/robustness/Mean_Guessing32.png}%
}%
\caption{Mean and maximum $L_1$ error of Smarts Ratings when a fraction of the $N$ players use Mean Guessing after $T$ time steps. In Mean Guessing, a player always gives Reported Guesses equal to the mean of all Smarts Ratings in the system. The error introduced by Mean Guessing depends linearly on the number of players using the strategy.}
\label{fig:MeanGuess}
\end{figure}

Another way that players may take advantage of a known distribution is via the Quantile Guessing strategy described above. The implementation of Quantile Guessing used for simulations uses the Empirical Cumulative Distribution Function with the sample composed of Actual Guesses of past opponents. A Reported Guess defaults to Actual Guesses if no Smarts Ratings have been determined, and for a player's first Luna Game. Ratings again collapse if all players use this strategy from the start, as shown in Figure \ref{fig:QuantileGuess}. 

\begin{figure}[h]
\centerline{%
\includegraphics[width=0.5\textwidth]{figures/robustness/Quantile_Guessing21.png}%
\includegraphics[width=0.5\textwidth] {figures/robustness/Quantile_Guessing22.png}%
}%
\caption{Mean and maximum $L_1$ error of Smarts Ratings when all $N$ players use Quantile Guessing. Quantile Guessing takes advantage of the distribution of all Smarts Ratings and a player's estimation of an opponent's rating percentile to formulate Reported Guesses. Due to the discrete nature of Smarts Ratings, Quantile Guessing causes Smarts Ratings to collapse to a constant, and the system cannot recover.}
\label{fig:QuantileGuess}
\end{figure}

If only a fraction of players use the strategy, the error depends linearly on the fraction, as was the case with Minimum and Mean Guessing. Figure \ref{fig:QuantileGuessFrac} confirms this result.

\begin{figure}[h]
\centerline{%
\includegraphics[width=0.5\textwidth]{figures/robustness/Quantile_Guessing31.png}%
\includegraphics[width=0.5\textwidth] {figures/robustness/Quantile_Guessing32.png}%
}%
\caption{Mean and maximum $L_1$ error of Smarts Ratings when a fraction of $N$ players use Quantile Guessing after $T$ time steps. Quantile Guessing takes advantage of the distribution of all Smarts Ratings and a player's estimation of an opponent's rating percentile to formulate Reported Guesses. The error introduced by Quantile Guessing depends linearly on the number of players using the strategy.}
\label{fig:QuantileGuessFrac}
\end{figure}

What if the distribution is revealed once a certain number of games have been played? One might hope that the collapsing to a constant problem is avoided in this case. To test this hypothesis, the simulation in Figure \ref{fig:QuantileGuessDelay} introduces players that play with honest guessing for a variable number of time steps, and then play with Quantile Guessing for $500$ time steps. The delay in introducing Quantile Guessing does indeed improve the $L_1$ error within the time frame tested, though significant error remains after $500$ time steps for all tested delays. 

\begin{figure}[h]
\centerline{%
\includegraphics[width=0.5\textwidth]{figures/robustness/Delayed_Quantile_Guessing41.png}%
\includegraphics[width=0.5\textwidth] {figures/robustness/Delayed_Quantile_Guessing42.png}%
}%
\caption{Mean and maximum $L_1$ error of Smarts Ratings when $N$ players give honest guesses for some number of delay time steps, and then shift to the Quantile Guessing strategy for 500 additional time steps. The shift delay is negatively correlated with Quantile Guessing, suggesting that the negative impact of the strategy decreases if it is introduced after LRS has already stabilized.}
\label{fig:QuantileGuessDelay}
\end{figure}

\subsection{Combined Strategies}

In a real instantiation of LRS, it is unlikely that any single strategy will be adopted by all players. Thus the extreme results portrayed above are not of urgent concern. More likely is a situation where a small fraction of players adopt each of the strategies above. For the purpose of simulation, I assume that some fraction of players will play honestly, and the remaining players will be evenly divided among the strategies described above: Random Guessing, Minimum Guessing, Mean Guessing, and Quantile Guessing. Figure \ref{fig:Combined} shows that the relationship between error and honest player proportion is linear, as might be expected given the previous simulations. Roughly, for every additional $10$\% of players that forgo honest guessing in favor of one of the other strategies, the mean $L_1$ error increases by $3.5$.

\begin{figure}[h]
\centerline{%
\includegraphics[width=0.5\textwidth]{figures/robustness/Combined_Dishonest_Strategies31.png}
\includegraphics[width=0.5\textwidth] {figures/robustness/Combined_Dishonest_Strategies32.png}%
}%
\caption{Mean and maximum $L_1$ error of Smarts Ratings when some fraction of $N$ players give honest guesses, and the remaining players are split evenly among Random Guessing, Minimum Guessing, Mean Guessing, and Quantile Guessing strategies. The relationship between error and honest player proportion is linear.}
\label{fig:Combined}
\end{figure}
\section{Implications for LRS Design}

The theoretical and simulation results discussed in this chapter provide a cautionary tale for LRS designers. The system should be successful if all players practice honest play, but other strategies, such as Random Guessing, Minimum Guessing, Mean Guessing, and Quantile Guessing, threaten to undermine the validity of Smarts Ratings. Fortunately, these strategies can be grouped into two categories: strategies that are unlikely, and strategies that can be prevented. 

Random Guessing and Minimum Guessing belong in the unlikely strategy group. Random Guessing would be a bizarre long-term choice; the strategy gives a player no advantage in either Smarts Rating or Luna Game outcomes. Minimum Guessing is also unlikely for the player who realizes the personal benefit in terms of relative Smarts Ratings is very small. Minimum Guessing can also claim membership to the preventable strategy group; players practicing this strategy can be quickly identified by the system and banned for violating the spirit of play. Since they rely on the distribution of Smarts Ratings, Mean Guessing and Quantile Guessing are also easily avoidable; the distribution can simply be withheld from players. The distribution may also be withheld initially and introduced once the system has evolved and stabilized. This delay would improve Quantile Guessing for players, and these later players would introduce less error to Smarts Ratings by using the strategy.

If the Smarts Rating distribution is withheld from players, designers must choose another method for describing the ratings, or else the players will have no basis to form their first Guesses. This chapter has discussed several avenues that should be avoided: fixed initializations of Smarts Ratings; a first-time ``quiz'' to initialize Smarts Ratings; or communicating only the average Smarts Rating. One option is to provide a vague disclaimer, such as ``Smarts Ratings are positive real numbers that typically range between $0$ and $100$.'' A more concrete alternative would be to ask players, ``On a scale from $0$ to $100$, how smart is your opponent?'' Of course, providing any numbers in the instructions encourages machine players to adopt Means Guessing or a similar naive strategy early on. Initializing these early Smarts Ratings is ultimately a chicken-or-egg problem that can only be resolved in practice by well-meaning human players.

The results in this chapter underscore the importance of good faith among players of the Luna Game. For researchers hoping to use LRS as a test of AI, the integrity of the system is essential if their own results are to be meaningful. Honest play is in the best interest of these players. Casual human players should also understand that dishonest strategies in mass can threaten both the validity and the distribution of Smarts Ratings, rendering LRS neither informative nor fun for players. The spirit of the Game should be made as clear to players as the limited attention span of Internet users permits. As with any test of human intelligence, LRS ultimately requires judges and subjects to play along.

%\include{chapters/conclusion}
%\begin{appendices}
%    \include{chapters/appendixA}
%\end{appendices}
%
%\setstretch{1.2}
%
% the back matter
\clearpage
\bibliography{VVT_Thesis_References}
\addcontentsline{toc}{chapter}{References}
\bibliographystyle{apalike2}

%\include{endmatter/colophon}

\end{document}
