% the abstract

The easy reprogrammability of the Cas9 RNA-guided endonuclease has made it one of the most versatile tools for genome editing. The protein's targeting activity occurs in three phases. First, it complexes with a guide RNA (gRNA), comprising of a spacer sequence and a double hairpin. Second, it scans DNA for a short Protospacer Adjacent Motif (PAM) sequence, which is specified by the protein itself. Third, when a PAM is found, Cas9 checks the gRNA spacer sequence for complementarity against the adjacent genomic DNA. If there is a match, Cas9 will cleave the DNA at that location. Though changing the spacer sequence on the gRNA reprograms the sequence specificity of Cas9, there is a 
natural limitation on its set of genomic targets due to the PAM sequence requirement. The canonical Cas9 used by most scientists, the \textit{S. pyogenes} Cas9 (SpCas9), has a PAM sequence of NGG, limiting it to $\approx$1 in 8 loci in the human genome. We recently characterized the PAM specificity of evolved Cas9 variants, as well as the \textit{S. aureus} Cas9 (Kleinstiver \textit{et al.} 2015), showing that that Cas9 proteins from orthogonal bacterial strains can indeed be characterized. Here, we propose to attempt to characterize the PAM specificities of 5 more Cas9 proteins from a variety of bacterial strains.
